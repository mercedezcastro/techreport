\documentclass[english]{proposalnsf}
\usepackage{graphicx}
\usepackage{url}
\usepackage[square,numbers]{natbib}
\usepackage[nottoc,numbib]{tocbibind}

\title{Undergraduates and Extra-curricular Achievements in CS}
\author{Mercedez Castro \\Collaborative Software Development Laboratory \\ Information and Computer Sciences \\ University of Hawaii}

\begin{document}
  \maketitle
  \tableofcontents
  \newpage

  \section{Introduction}
  \label{introduction}

  Extra-curricular is no longer extra.
  Computer science has an overwhelmingly large problem with undergraduate participation in extra-curricular activities.
  Whether it be in the industry or graduate school, successful graduates do not have internships, meet-ups, and hackathons as part of their undergraduate education.
  In modern computer science education, a well prepared computer science undergraduate must have extra-curricular activities as part of their degree experience.
  The traditional undergraduate department cannot keep up with the ever changing technologies and opportunities that they produce.
  Despite the fact that extra-curricular activities are so important, an alarming number of undergraduates do not take advantage of the various opportunities that they have at their disposal.
  Why is this?\\*

  In order to understand how to increase participation of undergraduates in extra-curricular activities, we must first begin to understand why undergraduates might not even try to participate in such opportunities.
  What are student attitudes that lead to the lack of participation in extra-curricular activities.
  In this research project, I will investigate what students believe to be barriers to completing extra-curricular achievements and try to discover whether there are attitudes that potentially underlay this problem.\\*

  My Hypothesis is that majority of students don't participate in extra-curricular activities for 2 reasons:
  \begin{itemize}
    \item Students believe that GPA is the only important factor
    \item Students don't realize the importance of extra-curricular achievements\\*
  \end{itemize}

  I believe that there is a positive correlation between ICS students who care and don't care about extra-curricular activities and their usage of the RadGrad system.
  If ICS students who currently do not care about extra-curricular activities experience the follow process:
  \begin{itemize}
    \item Student is informed about the importance of extra-curricular activities in achieving their specific goal
    \item Student outlines a plan for the next semester on the RadGrad system to become involved in extracurricular activities they are interested in,
  \end{itemize}
  then they will become involved in extracurricular activities for the following semester.

  \section{Related Work}
  \label{related-work}

  TBD

  \section{Research Design}
  \label{research-design}

  In order to address the problem of ICS undergraduate participation in extra-curricular achievements, I will first conduct interviews that ask reflective questions and introduce participants to the abundance of resources available on the RadGrad system.
  In turn, I believe that at least 50\% of participants will have completed at least 1 extra-curricular activity by the end of the experiment (Spring 2020).\\*

  \textbf{Reflective Questions}\\*

  Weighted Questions
  \begin{itemize}
    \item How important do you believe GPA is to prospective recruiters?
    \item How much pressure do you feel to achieve a high GPA?
    \item How important do you believe extra-curricular achievements are to prospective recruiters?
    \item How much pressure do you feel to participate in extra-curricular activities?
    \item How appealing do you believe your resume is to prospective recruiters?\\*
  \end{itemize}

  Open-ended Questions
  \begin{itemize}
    \item Are you a full time or part time student?
    \item Are you currently working?
    \item Is it an ICS related job?
    \item Have you participated in any extra-curricular activities?
    \item What enticed you to participate in this event?
    \item What were the main takeaways from that experience?
    \item Do you believe this experience will help you in a future career?
    \item Would you recommend it to other students/participate in that event again?
    \item Has any advisor/mentor/professor spoke to you about potential ICS related opportunities?
    \item Do you feel that you have an adequate amount of time to study in order to perform well academically?
    \item How much more time (hours) would be ideal?
    \item Do you have any side interests/hobbies?
    \item Do you think you have enough free time to work on side hobbies/interests?
    \item What type of job do you hope to get when you graduate?
    \item How do you plan on obtaining such job?
    \item Have you attended a career fair in the past?
    \item Have you attended a resume workshop in the past?
    \item What did you think about your experience?
    \item Have you used RadGrad?
    \item What do you think RadGrads intended purpose is?
    \item What do you believe recruiters look for most in a resume?
    \item What about second most?
    \item Regarding project experience on a resume, what do you think is the ideal amount to include (if any)?\\*
  \end{itemize}

  RadGrad Onboarding Questions
  \begin{itemize}
    \item Which opportunity are you most interested in?
    \item Which opportunity do you think you are most interested in that will earn you the most amount of Innovation or Experience points?\\*
  \end{itemize}

  Post-Interview Questions (@ end of Spring 2020)
  \begin{itemize}
    \item Did you end up participating in any of the RadGrad opportunities that you were previously interested in?
    \item Did you attend any resume workshops or career fairs?\\*
  \end{itemize}

  By showing a positive correlation between RadGrad usage and caring about extracurriculars we can investigate the claim that the causation is:
  Students don't care --> Don't use RadGrad
  and demonstrate evidence that RadGrad is effective at making students care about extracurricular activities.

  \section{Data Collection}
  \label{data-collection}

  \begin{itemize}
    \item Name
    \item Gender
    \item Major
    \item Class standing
    \item Amount of credits currently taking
    \item Expected graduation date
    \item GPA ( < 3.0 or > 3.0 )
    \item Work experience
    \item Extra-curricular activities
    \item RadGrad usage
  \end{itemize}

  I suspect that there will be a significant difference in GPA and extracurricular perspectives depending on class standing, graduation date, GPA, and work experience.

  \section{Deliverables}
  \label{deliverables}

  1st Interview
  \begin{itemize}
    \item Collect data (recorded)
    \item Note and organize data into prospective categories
    \item See if there is any correlation between class standing/GPA/expected graduation date and extra-curricular participation. \\
  \end{itemize}

  2nd Follow-up Interview
  \begin{itemize}
    \item See if a positive correlation exists between being informed/caring and RadGrad usage.
    \item Gather experiment results
  \end{itemize}

  \section{Concerns}
  \label{questions-concerns}

  \begin{itemize}
    \item Do participants need to sign permission forms, etc.?
    \item What will my sample size be? Will it be enough?
    \item Are there certain demographics I should focus on?
    \item Will participants be accurate  in answering questions?
    \item How do I access the accuracy of their answers?
    \item How they truly feel vs how they think they are expected to feel
  \end{itemize}

  \section{Results}
  \label{results}

  TBD

  \section{Conclusions}
  \label{conclusions}

  TBD

  \bibliography{techreport}
  \bibliographystyle{plainnat}

  \appendix
\end{document}

